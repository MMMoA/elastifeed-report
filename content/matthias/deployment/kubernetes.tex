\subsection{Einführung in Kubernetes}

Ein \textit{\ac{k8s} Cluster} ermöglicht die automatisierte Bereitstellung von Containern für etwa eine Microservice-Architektur.
K8s besteht aus zwei Komponenten, die im Cluster mindestens einmal vorkommen: \textit{Controller} und \textit{Kubelet}, die auf physischen oder virtuellen Servern installiert werden.
Die Kubelet's verwalten Container auf einem Server und werden von der Controller Komponente angesteuert.
Server auf denen von Kubelet Container verwaltet, werden als \textit{Compute Node} bezeichnet.
Der Controller nimmt über eine HTTP-basierte Schnittstelle Konfigurationen entgegen und verteilt Container entsprechend der Konfiguration auf den Compute Nodes.
Die Konfiguration erfolgt über YAML-Dateien, die mit dem Kommandozeilen-Programm \texttt{kubectl} an einen Controller übermittelt werden.

\subsubsection{Pod}
Die kleinste Einheit von \ac{k8s} bildet der \textit{Pod}.
Er kann einen oder mehrere Container sowie Daten enthalten.
Pods können erstellt werden, werden aber nicht von \ac{k8s} überwacht.
Sollte ein Pod nicht verfügbar sein, unternimmt \ac{k8s} keine weiteren Aktionen.

\subsubsection{Deployment}
Um die Verfügbarkeit eines Pod zu Garantieren, kann ein \textit{Deloyment} verwendet werden.
Dieses beschreibt, wie der zu erzeugende Pod aussehen soll und wie viele Replikate davon existieren sollen.
Der Controller von K8s verteilt die zu erzeugenden Pods auf die Compute Nodes.
Wenn jetzt ein Pod abstürzt oder der Server auf dem der Pod ausgeführt wird abstürzt, erkennt der Controller den Ausfall und initiiert die Bereitstellung eines Ersatzes.

\subsubsection{StatefulSet}
Pods und Deployments sind stateless und können keine Daten speichern, die nach Neustart oder Löschen eines Pods weiter existieren.
Um auch Komponenten die Daten speichern müssen, in \ac{k8s} zu betreiben existiert das \textit{StatefulSet}.
Dies ist genauso aufgebaut wie ein Deployment, kann aber zusätzlich Speicher bereitstellen.
Wenn ein Pod ausfällt, wird ein Neuer erzeugt, der auf den gleichen persistenten Speicherbereich wie der Alte zugreifen kann.
Beispielsweise Datenbanken die in einem \ac{k8s} Cluster laufen, müssen auf ein StatefulSet zugreifen.

\subsubsection{Service}
Innerhalb des Clusters kann auf die einzelnen Pods zugegriffen werden.
Um einzelne Komponenten einer Microservice Architektur zusammenzufassen und die Skalierbarkeit zu gewährleisten bietet \ac{k8s} einen \textit{Service} an.
Dieser ist innerhalb des Clusters erreichbar und verteilt eingehende Anfragen auf alle Pods die dem Service zugeordnet sind.

\subsubsection{Ingress}
Um von außen über HTTP auf Komponenten innerhalb des Clusters zugreifen zu können, kann ein \textit{Ingress} genutzt werden.
Dieser leitet eingehende Anfragen, die auf ein definiertes Muster passen an einen \ac{k8s} Service weiter.
