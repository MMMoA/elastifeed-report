\subsection{Elasticsearch}
Die Elasticsearch-Datenbank kann als Cluster aufgesetzt werden, der vier unterschiedliche Anwendungsarten beinhalten kann:
\begin{enumerate}
        \item \textbf{data node}: Hier werden eigentliche Daten gespeichert und Operationen auf den Daten ausgeführt
        \item \textbf{master node}: Zuständig für die Cluster Verwaltung
        \item \textbf{clien node}: Leitet Anfragen an data oder master Nodes weiter
        \item \textbf{Ingest node}: Transformieren von Dokumenten bevor sie zu einem Index hinzugefügt werden
\end{enumerate}
Die minimale Konfiguration eines "Cluster" ist dabei eine Master und eine Daten Instanz - beide Dienste können von einer Instanz abgebildet werden.
Um die uns zur Verfügung stehenden Ressourcen so effektiv wie möglich zu nutzen, nutzen wir auf den 3 VMs je eine Elasticsearch-Instanz im Master und Daten Modus.
Um ein Split Brain - zwei gegeneinandere operierende Master Nodes - zu vermeiden, ist der Cluster erst einsatzbereit sobald zwei oder mehr Nodes dem Cluster beigetreten sind.
Die Installation von Elasticsearch wird wie auch die Provisionierung der VMs von Ansible übernommen und kann auf beliebig viele Instanzen im Cluster skaliert werden.

Mit Kibana wird eine Web-Oberfläche für Elasticsearch angeboten, die wir insbesondere zum Debuggen häufig nutzen. Installiert wird diese erneut mit der Unterstützung von Ansible.
