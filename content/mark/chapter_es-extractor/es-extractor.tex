% !TeX encoding=utf8
% !TeX spellcheck = de-DE
\section{Implementierung} \label{extractor:sec:impl}
Mit dem Mercury Web Parser sind wir in der Lage zuverlässig den relevanten Text eines Artikels zu extrahieren.
Deshalb ist der \hyperref{https://github.com/elastifeed/es-extractor/}{}{}{es-extractor} wenig mehr als ein Wrapper für den Mercury Web Parser, der dessen Funktionen als REST Interface zur Verfügung stellt. Der es-extractor ist komplett mittels JavaScript's \mintinline{js}{async/await} Syntax zur asynchronen Programmierung implementiert. Dies bedeutet, dass verschiedene Aufgaben gleichzeitig abgearbeitet werden können. Beispielsweise muss eine zweite Anfrage nicht darauf warten, dass die Seite der ersten Anfrage geladen ist, bevor sie selber anfangen kann, zu laden. In unseren Tests können 2 bis 3 Anfragen gleichzeitig bearbeitet werden, ohne dass es zu spürbaren Leistungsverlusten kommt. \par
Eine erfolgreiche Abarbeitung einer Anfrage resultiert in einer HTTP 1.1 Antwort mit dem Statuscode 200. \verb|Content-Type| ist \verb|application/json|, mit UTF-8 Zeichenkodierung. Das Datum ist ein \ac{JSON} Objekt, welches dieselben Felder enthält wie die Antwort des Mercury Web Parser (Tabelle \ref{extractor:table:mercury}), mit Ausnahme des \verb|content| Feldes. Das \verb|content| Feld ist durch 2 Felder ersetzt worden: \\ \\
	\begin{tabu}{lX}
		\toprule
		Feld & Beschreibung \\ \midrule
		\texttt{raw\_content} & Relevanter Text des Artikels, komplett ohne Textformatierung. Dieses Feld sollte genutzt werden um Suchanfragen und ähnliches zu starten. \\
		\midrule
		\texttt{markdown\_content} & Relevanter Text, der als Markdown \cite{markdown2016} formatiert ist. Dieses Feld sollte verwendet werden, um den Text anzuzeigen, da er die ursprüngliche Form des Artikels beibehält. Er kann auch Links zu Bildern enthalten, die heruntergeladen werden müssen. \\
		\bottomrule
	\end{tabu}
\subsection{Schnittstellen}
Bei Programmstart wird standardmäßig ein HTTP Server auf \verb|http://localhost/|, Port 8080 gestartet. Dieser stellt 2 Endpunkte zu Verfügung, die beide nur auf einen HTTP 1.1 POST Request mit \verb|Content-Type: application/json| reagieren. 
\paragraph{/mercury/url} Dieser Endpunkt erwartet ein \ac{JSON} Objekt mit einem \quote{url} Feld. Die URL muss die URL der Website sein, von der der relevante Text bestimmt werden soll.
\mint{json}{{"url" : "http://example.com"}}
\paragraph{/mercury/html} Dieser Endpunkt erwartet ein \ac{JSON} Objekt mit einem \quote{url} Feld und einem \quote{html} Feld. Die URL muss die URL der Website sein, von der der relevante Text bestimmt werden soll und das HTML Feld muss dessen komplettes HTML enthalten.
\begin{minted}{json}
{
 "url"  : "http://example.com" , 
 "html" : "<html><head><title>Example<\title>..."
}
\end{minted}
\subsection{Logging}
Der es-extractor besitzt grundlegende Protokollierung, die zu \verb|stdout| geschrieben werden. Der Logger basiert auf dem \hyperref{https://www.npmjs.com/package/winston}{}{}{Winston Logger} für NPM, man kann ihm einen neuen \quote{Transport} hinzuzufügen, um dafür zu sorgen, dass die Protokolle von außen erreichbar sind. \\ 
Die Ausgabe enthält immer zuerst das Log - Level (info, warn, error), den Tag und die Uhrzeit. Eine Ausgabe, die durch eine Anfrage an den Server erzeugt wurde, beinhaltet zusätzlich eine ID, die diese Anfrage eindeutig identifiziert. Schlussendlich kommt die Nachricht.
\subsection{Fehlermeldungen}
Falls es während der Laufzeit zu einem Fehler kommt, wird dieser Fehler zuerst mit einem entsprechenden Level protokolliert. Eine Anfrage die einen Fehler erzeugt, bekommt grundsätzlich immer eine HTTP 1.1 Antwort mit einem Statuscode der ungleich 200 ist. In Fällen wo der Fehler beispielsweise durch eine inkorrete URL oder HTML Dokument erzeugt wird, wird zusätzlich ein JSON Dokument mit einer Nachricht zurückgeschickt.



















