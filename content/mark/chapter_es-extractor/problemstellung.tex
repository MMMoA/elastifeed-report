% !TeX encoding=utf8
% !TeX spellcheck = de-DE
\section{Problemstellung}
Wir haben schnell festgestellt, dass es sinnvoll ist, das Extrahieren von Text nicht nur für Artikel im Internet zu betrachten, sondern für alle Websites. Zu diesem Schluss sind wir gekommen, weil das jede Website im Internet unterschiedlich aufgebaut ist und man kaum einen Artikel von einer anderen Website unterscheiden kann. Indem wir die Funktion auf beliebige Seiten des Internets ausweiten machen wir die Verwendung des Systems aus Nutzersicht einfacher, da dieser sich nicht damit befassen muss, ob eine Seite als \quote{Artikel} zählt. Weiterhin sparen wir uns aus Sicht der Entwickler und Administratoren die Notwendigkeit zu definieren welche Seiten erlaubte \quote{Artikel} sind. Insbesondere gibt es potenziell unendliche viele zulässige \quote{Artikel}, alle einzutragen oder eine Regel dafür aufzustellen wäre vermutlich unmöglich.
Im Folgenden soll Artikel gleichbedeutend mit einer beliebigen Website, die Text enthält, sein.\\
Wir definieren Text eines Artikels als relevant, wenn er einen inhaltlichen Nutzen hat. Dies sind zum Beispiel Überschriften, Unterüberschriften, inhaltlicher Text, Listen und ähnliches. Wir wollen vermeiden Teile wie Navigationselemente, Werbung, Kommentare und Verlinkungen zu extrahieren. \\
Mit dem Kontext unserer Architektur lässt sich die Aufgabe weiter konkretisieren. Die Aufgabe der Textentnahme ist es also, einen Link entgegenzunehmen und für diesen zu bestimmen, welche Teile dieser Website relevant sind. Die Architektur sollte zu der Microservice – orientierten Architektur passen, heißt, es sollte ein REST Interface geben und das Ergebnis muss so formatiert sein, dass es auch wieder gut von anderen Microservices verwendbar ist. Es dürfen somit keine prozessspezifischen Adressen, Objekte oder Datenstrukturen zurückgegeben werden. \\ \par
Soweit die Grundlagen – ein REST Interface zu implementieren ist relativ simpel und benötigt keine weitere theoretische Diskussion. Wir werden es ausschließlich im \autoref{extractor:sec:impl} über den schlussendlichen Aufbau betrachten. Viel wichtiger ist jedoch die Frage, wie wir nun bestimmen können welcher Teil einer Website relevant ist. 

% @ TODO Add "Microservice" to Glossary?
