% !TeX encoding=utf8
% !TeX spellcheck = de-DE


\chapter{Textentnahme - es-extractor}
\chapterauthor{Mark Martinussen}

Eine der wichtigsten Anforderungen an das Projekt war eine Funktion, die den relevanten Text aus einem beliebigen Artikel zu extrahiert. In diesem Abschnitt werden wir zuerst zwei formale Ansätze betrachten und dann im Anschluss sehen wie wir diese Funktion umgesetzt haben.
% !TeX encoding=utf8
% !TeX spellcheck = de-DE
\section{Problemstellung}
Wir haben schnell festgestellt, dass es sinnvoll ist, das Extrahieren von Text nicht nur für Artikel im Internet zu betrachten, sondern für alle Websites. Zu diesem Schluss sind wir gekommen, weil das jede Website im Internet unterschiedlich aufgebaut ist und man kaum einen Artikel von einer anderen Website unterscheiden kann. Indem wir die Funktion auf beliebige Seiten des Internets ausweiten machen wir die Verwendung des Systems aus Nutzersicht einfacher, da dieser sich nicht damit befassen muss, ob eine Seite als \quote{Artikel} zählt. Weiterhin sparen wir uns aus Sicht der Entwickler und Administratoren die Notwendigkeit zu definieren welche Seiten erlaubte \quote{Artikel} sind. Insbesondere gibt es potenziell unendliche viele zulässige \quote{Artikel}, alle einzutragen oder eine Regel dafür aufzustellen wäre vermutlich unmöglich.
Im Folgenden soll Artikel gleichbedeutend mit einer beliebigen Website, die Text enthält, sein.\\
Wir definieren Text eines Artikels als relevant, wenn er einen inhaltlichen Nutzen hat. Dies sind zum Beispiel Überschriften, Unterüberschriften, inhaltlicher Text, Listen und ähnliches. Wir wollen vermeiden Teile wie Navigationselemente, Werbung, Kommentare und Verlinkungen zu extrahieren. \\
Mit dem Kontext unserer Architektur lässt sich die Aufgabe weiter konkretisieren. Die Aufgabe der Textentnahme ist es also, einen Link entgegenzunehmen und für diesen zu bestimmen, welche Teile dieser Website relevant sind. Die Architektur sollte zu der Microservice – orientierten Architektur passen, heißt, es sollte ein REST Interface geben und das Ergebnis muss so formatiert sein, dass es auch wieder gut von anderen Microservices verwendbar ist. Es dürfen somit keine prozessspezifischen Adressen, Objekte oder Datenstrukturen zurückgegeben werden. \\ \par
Soweit die Grundlagen – ein REST Interface zu implementieren ist relativ simpel und benötigt keine weitere theoretische Diskussion. Wir werden es ausschließlich im \autoref{extractor:sec:impl} über den schlussendlichen Aufbau betrachten. Viel wichtiger ist jedoch die Frage, wie wir nun bestimmen können welcher Teil einer Website relevant ist. 

% @ TODO Add "Microservice" to Glossary?

% !TeX encoding=utf8
% !TeX spellcheck = de-DE
\section{Theoretische Lösungsansätze}
\subsection[Die reduzierte Druckansicht]{Ansatz 1: Druckansicht}% \addcontentsline{toc}{subsection}{Ansatz 1: Druckansicht} % Add to table of contents without enumerating
\label{extractor:ansatz:subsec1}
Der erste Ansatz, den wir in Betracht gezogen haben, um den Inhalt eines Artikels zu extrahieren war die \quote{Druckansicht} der Seite zu verwenden. Hiermit ist diejenige Version des Artikels gemeint, welche erzeugt werden würde, wenn man mit einem Browser die Seite auf Papier ausdruckt. Wie uns aufgefallen ist, besitzen einige Seiten eine Druckansicht, die sich von der dargestellten Version unterscheiden, oft indem Elemente wie Navigation, Bilder und Werbung entfernt sind und fast nur der Inhalt dargestellt wird. Beispielsweise die News – Seite \url{www.heise.de} besitzt für all ihre Artikel eine solche Druckversion. (siehe \autoref{extractor:image:druckansicht})

\begin{figure}[t]
	\centering
	\includegraphics[width=\linewidth]{images/druckansicht.png}
	\caption{Ein Artikel und dessen reduzierte Druckversion}
	\label{extractor:image:druckansicht}
\end{figure}

Man sieht wie die vollständige Website auf der linken Seite Navigationselemente und Werbung, die wir als nicht relevant ansehen enthält. Rechts sind dahingegen nur die relevanten Inhalte zu sehen. Falls wir also in der Lage sind, zu einem solchen Artikel vom Link auf diese Druckansicht zu schließen, könnten wir diese verwenden, um den relevanten Text einer Website zu bestimmen. Um diesen Lösungsansatz zu implementieren muss man mit zwei zentralen Fragen beschäftigen.
\begin{enumerate}
	\item Wie erhält man die Druckansicht des Artikels, falls diese existiert?
	\item Kann man bestimmen ob eine Website überhaupt eine reduzierte Druckansicht hat? 
\end{enumerate}
Ein Werkzeug zur Lösung der ersten Frage sind \quote{headless} Webbrowser. Der Begriff \quote{headless} beschreibt hierbei, dass ein Programm, welches normalerweise mit einer grafischen Benutzeroberfläche ausgeführt wird, komplett ohne diese rein algorithmisch gesteuert wird. Wir werden in \autoref{scraper:sec:headless} sehen, wie man einen headless Browser verwendet, da wir für das Rendern von Bildern die gleiche Herausforderung haben werden. \\ \\
Aufgrund der zweiten Frage müssen wir leider feststellen, dass dieser Lösungsansatz nicht ausreichend ist. So gibt es keinen uns bekannten Weg, festzustellen ob ein Artikel eine reduzierte Druckansicht anbietet. Wie oben besprochen, ist jede Website unterschiedlich und es gibt keine fest definierte Schnittstelle. Weiterhin würde dieser Lösungsansatz eben auch nur bei Artikeln funktionieren, die eine solche Druckansicht anbieten, aber eben nicht, bei all jenen, die keine anbieten. Somit können wir unser Ziel mit diesem Ansatz nicht erreichen und müssen ihn leider verwerfen.


\subsection[HTML DOM]{Ansatz 2: HTML DOM} %\addcontentsline{toc}{subsection}{Ansatz 2: HTML DOM} 
\label{extractor:ansatz:subsec2}
Statt zu versuchen auf ein komplexes Feature, welches nur von bestimmten Seiten angeboten wird, zuzugreifen haben wir einen Ansatz entwickelt, welcher lediglich die HTML Datei einer Website benötigt. Eine jede Website wird durch eine HTML Datei beschrieben, die man ganz einfach mit dem Link herunterladen kann.
Diese Datei besitzt ein wohl definiertes Format. Die Idee dieses Ansatzes ist es nun, sich ebenjenes Format, und insbesondere die Meta – Informationen, die es enthält, zu nutzen zu machen, um den relevanten Text zu bestimmen. \\ \\ 
Der Aufbau einer HTML Datei oder HTML Dokument, wird durch das \ac{HTML DOM} beschrieben: Ein Dokument besteht aus mehreren Elementen, wobei ein Element mit den spitzen Klammern \mintinline{html}{<name>} eingeleitet und mit \mintinline{html}{</name>} terminiert wird. Im Sinne des \ac{HTML DOM} sind diese Elemente Objekte. Das heißt, sie besitzen Eigenschaften (Attribute), Methoden und Ereignisse. \cite{w3c_html} Elemente die geschachtelt definiert sind, bilden eine Vererbungsbeziehung, wobei das oberste Objekt immer \verb|document| genannt wird. Ein HTML Dokument beschreibt einen Baum (\autoref{extractor:image:html}), mit einem \verb|head| für Metainformationen und einem \verb|body| für den Inhalt. Dieser Inhalt kann dann weiter in Absätze, Überschriften, usw. unterteilt sein.  
\begin{wrapfigure}{r}{0.4\textwidth}
	\centering
	\includegraphics[scale=0.7]{images/html_overview.png}	
	\caption{Struktur eines HTML Dokumentes}
	\label{extractor:image:html}
\end{wrapfigure}

Nun können wir die Eigenschaften dieser Objekte auszunutzen. Beispielsweise gibt es das Attribut \mintinline{js}{document.body.textContent}, das allen angezeigten Text des Artikels enthält. Dieses beinhaltet jedoch auch den Text der Navigationselemente und Werbung. Folglich ist dieses Attribut alleine nicht ausreichend.

\subsubsection*{CSS Query Selectors}
Zusätzlich zu seinen Attributen kann ein Element auch mit einer Klasse und einer ID dekoriert werden. Diese sind nicht direkt Teil von HTML, aber stattdessen Teil des CSS Stylings. So kann ein Webentwickler bestimmen, dass der Absatz, der den Haupttext enthält, eine bestimmte ID hat oder Teil einer Klasse ist. IDs, Klassen und HTML Elemente können dann mit sogenannten CSS Query Selectors abgefragt werden. Diesen Umstand können wir uns zunutze machen und selber diese Selektoren ausführen, um an die gewünschten Elemente zu kommen. Ein Auszug aus der Syntax Definition:
\begin{itemize}
	\item \mintinline{html}{element element2} Wählt alle Elemente aus, deren Vorfahre ein HTML Element \quote{element} ist, und die selber ein Element vom Typ \quote{element2} sind.
	\item \mintinline{html}{element,element2} Wählt Elemente aus, die entweder element oder element2 sind.
	\item \mintinline{html}{.class} Wählt alle Elemente aus, deren Klasse \quote{class} ist.
	\item \mintinline{html}{#id} Wählt alle Elemente aus, deren ID gleich \quote{id} ist.
	\item \mintinline{html}{[attribut]} Wählt alle Elemente aus, die ein solche Atrribut besitzen.
\end{itemize}
Nun sind wir in der Lage, diese Query Selectors anzuwenden um den relevanten Text eines Artikels zu bestimmen. Beispielsweise können wir mit der Zeile 
\begin{minted}{js}
document.body.querySelectorAll(
	'.a-article-header__title,.a-article-header__lead,.article-content')
\end{minted}
die relevanten HTML ELemente eines beliebigen Artikels von \url{www.heise.de} extrahieren. Die Formatierung bleibt hierbei erhalten, heißt wir können sogar Überschriften, Zitate, Aufzählungen und Tabellen unterscheiden und später unverändert anzeigen. Um den reinen relevanten Text zu erhalten würden wir von jedem Listenelement das \verb|textContent| Attribut verwenden.
























% !TeX encoding=utf8
% !TeX spellcheck = de-DE
\section{Mercury Web Parser}
Wie wir in Abschnitt \ref{extractor:ansatz:subsec2} besprochen haben, können wir mithilfe des \ac{HTML DOM} und CSS Query Selectors zuverlässig den relevanten Text  eines Artikels bestimmen. Damit wir nicht Selektoren für alle Websites der Welt definieren müssen, verwenden wir eine Library, die uns viel Arbeit abnimmt. Der Mercury Web Parser von Postlight Labs LLC\cite{mercury_homepage} verwendet genau eine Menge von komplexen CSS Query Selectors und Eigenschaften des HTML DOM um für einen beliebigen Artikel den relevanten Text und mehr zu bestimmen. Der Mercury Parser kann mit nur einer URL oder zusätzlich dem schon heruntergeladenen HTML Dokument aufgerufen werden. In der Praxis verwenden rufen wir ihn nur mit der URL auf. Zusätzlich definieren wir den HTTP Header der Anfrage und mit welcher Formatierung (keine, HTML, Markdown) der relevante Text extrahiert werden soll.

\begin{listing}[ht]
	\label{extractor:code:mercury}
	\inputminted[
		frame=lines, 
		framesep=2mm, 
		linenos, 
		baselinestretch=1.2, 
		fontsize=\footnotesize]{js}{code/mercury-example.js}
	\caption{Beispielhafter Aufruf des Mercury Web Parsers}
\end{listing}


Das Ergebnis ist ein \ac{JSON} Objekt, die enthaltenen Felder sind in Tabelle \ref{extractor:table:mercury} aufgelistet. Falls der Parser den Wert für ein Feld nicht bestimmen kann, ist er \verb|null|.

\begin{table}
	\centering
	\begin{tabu}{lX}
		\toprule
		Feld & Beschreibung \\ \midrule
		\texttt{title} & Titel oder Überschrift der Website. \\
		\texttt{content} & (Formatierter) Text, welcher als relevant angesehen wird. Mögliche Formate sind HTML, Markdown oder keins. \\
		\texttt{author} & Autor oder Editor der Seite / Artikels. \\
		\texttt{date\_published} & Zeitpunkt, zu dem die Website veröffentlicht wurde. \\
		\texttt{lead\_image\_url} & URL des Vorschaubildes.  \\
		\texttt{dek} & Abstrakte Zusammenfassung des Artikels, die unter dem Titel steht. \\
		\texttt{excerpt} & Ein kleiner Ausschnitt des Artikels, oft die ersten paar Sätze oder identisch mit dem \texttt{dek}. \\
		\texttt{total\_pages} & Anzahl der Seiten (HTML Dokumente) über die sich dieser Artikel erstreckt.\\
		\texttt{next\_page\_url} & URL der nächsten Seite (HTML Dokument) des Artikels, falls der Artikel aus mehreren Seiten besteht. \\
		\texttt{rendered\_pages} & Anzahl der Seite (HTML Dokumente), die betrachtet wurden. \\
		\texttt{url} & URL der Website. \\
		\texttt{domain} & Domain der Website. \\
		\texttt{word\_count} & Anzahl der Wörter im relevanten Text auf der Seite. \\
		\texttt{direction} & Vermutete Leserichtung des Artikels. Meistens \quote{ltr} oder \quote{rtl}. \\ \bottomrule
		
	\end{tabu}
\caption{Rückgabewerte des Mercury Web Parsers}
\label{extractor:table:mercury}
\end{table}
\begin{listing}
	\inputminted[
	frame=lines, 
	framesep=2mm, 
	linenos, 
	baselinestretch=1.1, 
	fontsize=\footnotesize]{js}{code/golemextractor.js}
	\label{extractor:code:golemext}
	\caption{Definition eines benutzerdefinierten Selektors}
\end{listing}

\subsection{Benutzerdefinierte Selektoren}
Trotz der fortlaufenden Arbeit, die an dem Mercury Web Parser gemacht wird, haben wir feststellen müssen, dass er bei weitem nicht perfekt ist. Insbesondere, da er von einem englischsprachigen Team entwickelt wird, werden deutschsprachige Websites und Artikel oft nur unzureichend analysiert. Eine HTML Datei mit deutschen IDs und Klassen wie \verb|.mitte| oder \verb|#inhalt| ist für den Parser unverständlich. Glücklicherweise sind die Entwickler sich dieses Problems bewusst und bieten eine Lösung an. \\
Es gibt die Möglichkeit, für eine Domain die unzureichend erkannt wird, benutzerdefinierte Selektoren anzulegen. Intern heißen diese \quote{Extraktoren}, wir werden diesen Begriff jedoch nicht verwenden, um Missverständnisse vorzubeugen. Bei diesen Selektoren handelt es sich ein \ac{JSON} Dokument, bei dem jedem Feld eine Menge von CSS Query Selectors zugeordnet wird. \cite{mercury_custom} Beispielsweise ein benutzerdefinierter Selektor für die deutsche News – Seite \url{www.golem.de}.















% !TeX encoding=utf8
% !TeX spellcheck = de-DE
\section{Implementierung} \label{extractor:sec:impl}
Mit dem Mercury Web Parser sind wir in der Lage zuverlässig den relevanten Text eines Artikels zu extrahieren.
Deshalb ist der \hyperref{https://github.com/elastifeed/es-extractor/}{}{}{es-extractor} wenig mehr als ein Wrapper für den Mercury Web Parser, der dessen Funktionen als REST Interface zur Verfügung stellt. Der es-extractor ist komplett mittels JavaScript's \mintinline{js}{async/await} Syntax zur asynchronen Programmierung implementiert. Dies bedeutet, dass verschiedene Aufgaben gleichzeitig abgearbeitet werden können. Beispielsweise muss eine zweite Anfrage nicht darauf warten, dass die Seite der ersten Anfrage geladen ist, bevor sie selber anfangen kann, zu laden. In unseren Tests können 2 bis 3 Anfragen gleichzeitig bearbeitet werden, ohne dass es zu spürbaren Leistungsverlusten kommt. \par
Eine erfolgreiche Abarbeitung einer Anfrage resultiert in einer HTTP 1.1 Antwort mit dem Statuscode 200. \verb|Content-Type| ist \verb|application/json|, mit UTF-8 Zeichenkodierung. Das Datum ist ein \ac{JSON} Objekt, welches dieselben Felder enthält wie die Antwort des Mercury Web Parser (Tabelle \ref{extractor:table:mercury}), mit Ausnahme des \verb|content| Feldes. Das \verb|content| Feld ist durch 2 Felder ersetzt worden: \\ \\
	\begin{tabu}{lX}
		\toprule
		Feld & Beschreibung \\ \midrule
		\texttt{raw\_content} & Relevanter Text des Artikels, komplett ohne Textformatierung. Dieses Feld sollte genutzt werden um Suchanfragen und ähnliches zu starten. \\
		\midrule
		\texttt{markdown\_content} & Relevanter Text, der als Markdown \cite{markdown2016} formatiert ist. Dieses Feld sollte verwendet werden, um den Text anzuzeigen, da er die ursprüngliche Form des Artikels beibehält. Er kann auch Links zu Bildern enthalten, die heruntergeladen werden müssen. \\
		\bottomrule
	\end{tabu}
\subsection{Schnittstellen}
Bei Programmstart wird standardmäßig ein HTTP Server auf \verb|http://localhost/|, Port 8080 gestartet. Dieser stellt 2 Endpunkte zu Verfügung, die beide nur auf einen HTTP 1.1 POST Request mit \verb|Content-Type: application/json| reagieren. 
\paragraph{/mercury/url} Dieser Endpunkt erwartet ein \ac{JSON} Objekt mit einem \quote{url} Feld. Die URL muss die URL der Website sein, von der der relevante Text bestimmt werden soll.
\mint{json}{{"url" : "http://example.com"}}
\paragraph{/mercury/html} Dieser Endpunkt erwartet ein \ac{JSON} Objekt mit einem \quote{url} Feld und einem \quote{html} Feld. Die URL muss die URL der Website sein, von der der relevante Text bestimmt werden soll und das HTML Feld muss dessen komplettes HTML enthalten.
\begin{minted}{json}
{
 "url"  : "http://example.com" , 
 "html" : "<html><head><title>Example<\title>..."
}
\end{minted}
\subsection{Logging}
Der es-extractor besitzt grundlegende Protokollierung, die zu \verb|stdout| geschrieben werden. Der Logger basiert auf dem \hyperref{https://www.npmjs.com/package/winston}{}{}{Winston Logger} für NPM, man kann ihm einen neuen \quote{Transport} hinzuzufügen, um dafür zu sorgen, dass die Protokolle von außen erreichbar sind. \\ 
Die Ausgabe enthält immer zuerst das Log - Level (info, warn, error), den Tag und die Uhrzeit. Eine Ausgabe, die durch eine Anfrage an den Server erzeugt wurde, beinhaltet zusätzlich eine ID, die diese Anfrage eindeutig identifiziert. Schlussendlich kommt die Nachricht.
\subsection{Fehlermeldungen}
Falls es während der Laufzeit zu einem Fehler kommt, wird dieser Fehler zuerst mit einem entsprechenden Level protokolliert. Eine Anfrage die einen Fehler erzeugt, bekommt grundsätzlich immer eine HTTP 1.1 Antwort mit einem Statuscode der ungleich 200 ist. In Fällen wo der Fehler beispielsweise durch eine inkorrete URL oder HTML Dokument erzeugt wird, wird zusätzlich ein JSON Dokument mit einer Nachricht zurückgeschickt.





















\chapter{Rendern von Websites - es-scraper}
\chapterauthor{Mark Martinussen}

In diesem Abschnitt werden wir sehen, wie wir aus einer HTML Datei PDF und JPG Dateien generieren können. Zusätzlich besprechen wir, wie wir diese rechenintensiven Aufgaben so effektiv wie möglich bearbeiten können, was uns dann schließlich zu der Implementierung des es-scraper bringen wird. 
% !TeX encoding=utf8
% !TeX spellcheck = de-DE
\section{Problemstellung}
Ergänzend zum Auslesen des relevanten Textes einer Website muss diese ebenfalls in Bildform angezeigt werden. Speziell wurde definiert, dass jede Seite in Form einer PDF und eines Screenshots vorlegen muss. \\
Dazu müssen wir bestimmen, wie man eine HTML Datei in solche Formate umwandelt. Wie benötigen also eine Art HTML Parser. Am bekanntesten sind HTML Parser in Webbrowsern, wo sie jeden Tag zum Einsatz kommen. Es stellt sich also die Frage, ob es einen Webbrowser gibt, der ohne eine grafische Benutzeroberfläche komplett über eine API oder als Library steuerbar ist.
Weiter haben wir bedacht, dass diese Operationen rechenintensiv sind. Demnach müssen wir einen Weg finden, die Implementierung möglichst effizient und nebenläufig zu machen. Das System darf auch bei einigen duzend Anfragen nicht hoffnungslos überlastet sein. \\
Selbstverständlich muss die Implementierung auch zu der Infrastruktur passen. Dementsprechend wird wieder ein REST Interface benötigt, die Applikation sollte nach der Philosophie der Microservices konzipiert sein und die Rückgabewerte sollten klar und einfach weiterzuverwenden sein.
% !TeX encoding=utf8
% !TeX spellcheck = de-DE
\section{Headless Webbrowser} \label{scraper:sec:headless}
Um ein HTML Dokument in eine PDF oder einen Screenshot umzuwandeln benötigen wir einen HTML Parser, der identisch zu einem Webbrowser arbeitet. So können wir sicherstellen, dass die archivierten Versionen, die wir den Benutzern anbieten mit den Originalen übereinstimmen. Dies beinhaltet nicht nur das Anzeigen von HTML, sondern auch das Ausführen von Javascript, Herunterladen von externen Bildern und ähnlichen Elementen sowie Anordnung aller Komponenten zu einem ganzen. \\
In dem Webbrowser Google Chrome gibt es die Möglichkeit eine Seite zu \quote{Inspizieren} (siehe \autoref{scraper:image:cdpinspect}). Dieses Instrument erlaubt dem Nutzer den Seiten zu bearbeiten und Probleme zu diagnostizieren. Intern heißen sie die Chrome DevTools. Mit einer langen Liste von Funktionen kann man sagen, dass die Chrome DevTools in der Lage sind, einen Webbrowser komplett zu emulieren:
\begin{itemize}
	\item Zugriff auf den HTML – Renderer von Chrome
	\item Sichtung und Modifikation des \ac{HTML DOM}
	\item Simulieren beliebiger Endgeräte
	\item Ausführung und Testen von Javascript
	\item Prüfen der Netzwerkaktivität
	\item Zugriff zu Cookies, Arbeitsspeicher und Websiteinterne Ressourcen
\end{itemize}


\begin{figure}[h]
	\centering
	\includegraphics[width=\linewidth]{images/cdp_inspect.png}
	\caption{Zugriff auf die Chrome DevTools in Google Chrome}
	\label{scraper:image:cdpinspect}
\end{figure}

Am wichtigsten für unseren Nutzen existiert zusätzlich eine Schnittstelle, mit dem man diese DevTools algorithmisch steuern kann. Das sogenannte \ac{CDP} ermöglicht kompletten Zugriff auf die Funktionen des Google Chrome Webbrowsers sowie des Chrome DevTools. Das \ac{CDP} stellt viele hundert Funktionen zur Verfügung die auf komplexe Weise zusammenarbeiten. \cite{cdpexplorer} 

\begin{figure}[h]
	\centering
	\includegraphics[]{images/cdp_overview.png}
	\caption{Struktur einer Chrome – Instanz}
	\label{scraper:image:cdptree}
\end{figure}
\autoref{scraper:image:cdptree} zeigt die Struktur einer Chrome Instanz. Das \quote{Browser} Element kann man sich als den übergeordneten Prozess vorstellen. Manchmal wird dieser auch der \quote{Allocator} genannt. Dieser Prozess besitzt dann etliche Ausführungskontexte, die größtenteils voneinander getrennt arbeiten können. Auf einem abstrakten Level kann man sie sich als verschiedene Fenster des Browsers vorstellen. Ein \quote{Fenster} hat, kann dann im Sinne von Tabs mehr als eine Seite anzeigen, welche selbst aus mindestens einem HTML Frame bestehen. \\
Wenn man mit dem \ac{CDP} entwickelt, ist es nützlich diese Struktur im Hinterkopf zu behalten. Man sollte wissen auf welcher Ebenen sich ein Funktionsaufruf auswirkt. Insbesondere können Aufrufe auch den Baum nach oben traversieren. Beispielsweise wenn ein man einen Mausklick auf einen Link im Frame simuliert, wirkt sich dieser selbstverständlich auch auf die Page aus.

%\pagebreak


Das \ac*{CDP} ist selbst in Javascript implementiert, Google empfiehlt dem normalen Entwickler es nicht direkt zu verwenden. \cite{dontuse} Für eine kleine Anwendung alle nötigen Funktionen korrekt zu verwenden sei schwer und fehleranfällig. Um auch nur einen simplen Webbrowser zu starten und zu navigieren müssen die richtigen Methoden in der richtigen Reihenfolge aufgerufen werden. Zuerst muss ein Browser erstellt oder das \ac{CDP} mit einem bestehenden Browser verbunden werden. Daraufhin müssen alle benötigten Kontexte korrekt konfiguriert werden, sodass eine Seite erstellt und navigiert werden kann. \\
Stattdessen wird geraten, vorbereitete Libraries zu verwenden, die diese komplexe Struktur abstrahieren und für die meisten Use – Cases ausreichend sind. 
\begin{table}[h]
	\centering
	\begin{tabu}{ll} \toprule
		Programmiersprache & Library \\ \midrule
		Node.js & Google Chrome Puppeteer \\
		Java & cdp4j \\
		Python & pychrome \\
		Go & chromedp, cdproto \\
		\bottomrule
	\end{tabu}
	\caption{Implementierungen des CDP}
	\label{scraper:table:cdpimpl}
\end{table}


Die obige Tabelle zeigt eine unvollständige Liste von Implementierung, die man in verschiedenen Programmiersprachen verwenden kann, um das CDP zu steuern. Wir zwei von diesen Libraries betrachten, wie wir sie verwenden können, welche Features sie bieten sowie einige Eigenartigkeiten, die uns im Projektverlauf aufgefallen sind.

\subsection*{Google Chrome Puppeteer} \label{scraper:subsec:puppeteer}
Google selbst schlägt als Abstraktion des \ac{CDP} die Verwendung ihrer eigenen Library vor. Google Chrome Puppeteer, kurz Puppeteer, ist eine Node.js API, welche die meisten Funktionen des \ac{CDP} implementiert um eine Chrome Instanz headless, oder mit einer Benutzeroberfläche zu steuern. \cite{puppeteer} Im Vergleich zur Verwendung des reinen \ac{CDP} ist Puppeteer einfach. So können wir einen neuen headless Browser starten und eine Website mit nur wenigen Zeilen laden.
\begin{minted}[frame=lines, 
framesep=2mm, 
linenos, 
baselinestretch=1.2, 
fontsize=\footnotesize]{js}
const puppeteer = require('puppeteer');

// Start a browser
const browser = await browser.launch(); 
// Create a page in the default context of this browser.
const page = await browser.newPage();  
// Navigate to a page
await page.goto('http://example.com/'); 
\end{minted}
In Hinblick auf \autoref{scraper:image:cdptree} ist auffällig, dass anscheinend kein BrowserContext erstellt werden muss. Es scheint, als ob eine Seite direkt in dem Hauptprozess / Allocator erstellt werden kann. Die ist jedoch nicht der Fall. Die Dokumentation von Puppeteer spezifiziert, dass alle Seiten in einem Standard – BrowserContext erstellt werden. Es ist zwar möglich mit der Funktion \mintinline{js}{browser.createIncognitoBrowserContext()} einen zusätzlichen Kontext anzulegen, dieser teilt aber keine Informationen mit den anderen Kontexten. Für besondere Fälle kann dies zu einem Problem werden und erfordert, dass man direkt mit dem CDP kommuniziert, um einen weiteren Kontext zu erstellen. Beispielsweise kann es sein, dass man in zwei Kontexten auf der gleichen Seite arbeiten möchte. Dies hat den Vorteil, dass man beide Kontexte parallel bearbeiten kann, während Informationen wie Cookies mit Login Daten zwischen den Kontexten geteilt werden. \footnote{Dies ist uns in der Implementierung passiert, da wir PDFs und Screenshots parallel rendern wollten.} \\
Nun können wir den HTML Renderer von Google Chrome nutzen um Screenshots und PDF Dateien herzustellen:  
\begin{minted}[frame=lines, 
framesep=2mm, 
linenos, 
baselinestretch=1.2, 
fontsize=\footnotesize]{js}
await page.screenshot({path : 'example.png'});
await page.pdf({path : 'example.pdf'});
\end{minted}
% !TeX encoding=utf8
% !TeX spellcheck = de-DE
\section{Entwicklung mit CDP in Go} \label{scraper:sec:chromedp}
Trotz der Zuverlässigkeit und Übersichtlichkeit von Google Chrome’s Puppeteer haben wir uns gegen die Verwendung dieser Library entschieden. Einerseits hatten wir keinen Entwickler im Team, der erfahren genug in der Verwendung von Node.js waren, um eine performante und stabile Implementierung zu entwickeln. Weiterhin haben wir in unseren ersten Tests Probleme mit der Nebenläufigkeit von Puppeteer gehabt. Wir konnten keinen Weg finden zuverlässig Screenshots und PDF Dateien einer Website gleichzeitig zu generieren. Da wir einen großen Fokus auf die Effizienz und Skalierbarkeit unserer Microservices legen, sind wir nicht in der Lage diese Library zu verwenden, obwohl es vermutlich die einfachste Variante ist. \\ \\
Die Programmiersprache Go bietet mächtige Primitive zur Entwicklung performanter, nebenläufiger Anwendungen. Es existiert ebenfalls eine direkte Implementierung des CDP in Go, bekannt als das Paket cdproto. Diese Library enthält generierten Code für alle des CDP definierten Befehle wie sie auch in der Referenzimplementierung zur Verfügung stehen. Diese Befehle zu generieren ist möglich, da das CDP nicht nur in Javascript implementiert ist, sondern auch vollständig mit PDL (Program Design Language) beschrieben ist. Aus dieser beschreibenden Sprache kann dann Quellcode für beliebige Programmiersprachen generiert werden.  \cite{cdprotogen}
Die Verwendung dieser Library ist immer noch sehr komplex. Auf der Grundlage des Pakets cdproto wird dann die Library chromedp entwickelt. Diese abstrahiert, ähnlich wie Puppeteer, die komplexen Funktionalitäten und vereinfacht die Nutzung. Ein großer Vorteil von der Implementierung in Go, dass die Standardbibliothek Kontexte ausgezeichnet unterstützt. \cite{go_context} Dadurch wird die Abbildung der Struktur einer Chrome Instanz sehr gut verwirklicht. Die Funktion \mint{go}{func NewContext(parent context.Context, opts ...ContextOption)} 
erstellt so je, nachdem mit welchem Vorgänger sie aufgerufen wird, einen Browser mit einem BrowserContext und einer Page oder einen BrowserContext mit einer Page in einem bestehenden Browser oder eine lediglich einen weiteren \quote{Tab} im selben BrowserContext.
\begin{listing}[h]
	\inputminted[
	frame=lines, 
	framesep=2mm, 
	linenos, 
	baselinestretch=1.2, 
	fontsize=\footnotesize]{go}{code/chromedp_context_1.go}
	\caption{Erstellung von geschachtelten Kontexten mit chromedp}
	\label{scraper:code:chromedpctx}
\end{listing}

\begin{listing}[h]
	\inputminted[
	frame=lines, 
	framesep=2mm, 
	linenos, 
	baselinestretch=1.2, 
	fontsize=\footnotesize]{go}{code/chromedp_context_2.go}
	\caption{Verwendung der Allocatorfunktionen in chromedp}
	\label{scraper:code:chromedpalloc}
\end{listing}

Wenn wir mehr Kontrolle über die Struktur unseres Browsers wünschen, können wir mit der Funktion \mint{go}{func NewExecAllocator(parent context.Context, opts ...ExecAllocatorOption)} nur den Allocator, also den Hauptprozess des Browsers, erzeugen. Mit diesem als Basis können wir nun beliebig viele BrowserContexts und Pages erstellen, indem wir wieder die Funktion \mintinline{go}{chromedp.NewContext(...)} verwenden. 


In dem Beispiel \ref{scraper:code:chromedpalloc} erschaffen wir nun also Schrittweise zuerst den Hauptprozess, dann zwei BrowserContext, welche man als \quote{Browser Fenster} verstehen kann. Gefolgt davon erzeugen wir 2 Tabs im ersten Fenster und einen Tab im zweiten Fenster. Wichtig ist, dass bei jedem Aufruf von \mintinline{go}{chromedp.NewContext(...)} immer ein Standardtab erzeugt wird, der Aktionen ausführen kann. Deshalb besitzen wir in Listing \ref{scraper:code:chromedpalloc} zwei Tabs mehr. Unsere Empfehlung ist, diese der Verständlichkeit wegen zu ignorieren. 

Um eine Aktion auf mit einem Tab auszuführen, dient die \\
\mintinline{go}{chromedp.Run(ctx context.Context, actions ...Action)} Funktion, die eine Liste von Aktionen auf einem Kontext ausführt. 
\begin{minted}[
frame=lines, 
framesep=2mm, 
linenos, 
baselinestretch=1.2, 
fontsize=\footnotesize]{go}
tasks := chromedp.Tasks{
  chromedp.Navigate("http://example.com"),
  chromedp.CaptureScreenshot(&result)
}
chromedp.Run(tab1_1, tasks)
\end{minted}
\pagebreak
\subsection{Verbindung zu einem bestehenden Browser} \label{scraper:subsec:go:remote}
Das CDP ermöglicht nicht nur die Erzeugung eines neuen Webbrowsers, sondern kann sich auch an einen bestehenden Browser verbinden. Dies hat den großen Vorteil, dass es zwischen verschiedenen Systemen keine unerwarteten Probleme gibt, wenn verschiedene Versionen von Chrome, den CDP und Go verwendet werden. Beispielsweise kann man einen Chrome Browser in einem Docker Container ausführen und so sicherstellen, dass auf allen Systeme derselbe Browser läuft. Die Funktion \mint{go}{func NewRemoteAllocator(parent context.Context, url string)} verbindet das CDP in Go mit einem beliebigen Chrome Instanz an einer Url. Es wird dann eine Verbindung über das Websocket Protokoll hergestellt. Damit sich das CDP an eine Browser Instanz verbinden kann, muss diese den entsprechenden Port öffnen. Mit dem Aufruf \mint{bash}{chrome --headless --remote-debuggig-port=9222} wird ein headless Browser gestartet, an den sich das CDP über  \verb|ws://localhost:9222| verbinden kann.

\subsection{Navigieren von Tabs} \label{scraper:subsec:go:navigate}
Um eine Funktion aus der chromedp Library, die nicht ausreichend oder fehlend ist, zu ersetzen verwendet man den Typ \mintinline{go}{type ActionFunc func(context.Context) error}, um eine herkömmliche Funktion in einem Kontext ausführbar zu machen. So haben wir die Navigation mit den Funktionen aus cdproto neu definiert, um sie an unseren Nutzen anzupassen.

\begin{listing}[h]
	\inputminted[
	frame=lines, 
	framesep=2mm, 
	linenos, 
	baselinestretch=1.2, 
	fontsize=\footnotesize]{go}{code/chromedp_navigate.go}
	\caption{Neudefinition der Navigation in chromedp}
	\label{scraper:code:navigate}
\end{listing}
Zuerst verwenden wir das \verb|emulation| Modul des CDP, um sicherzustellen, dass der User Agent gesetzt ist. Ohne User Agent bekommen wir keinen Zugang zu manchen Websites, die Beispielsweise erwarten, dass ein Nutzer zuerst den Cookies zustimmt, bevor man weitergeleitet wird. Daraufhin navigieren wir den Tab und warten darauf, dass ein \verb|PageLoadEvent| ankommt. Dieses Event signalisiert, dass die Website vollständig geladen ist und der Tab ihren gesamten Inhalt anzeigt. Wenn zu viele Seiten parallel geladen werden, kann es zu einem Data Race kommen. Dies liegt daran, dass der Browser mehrere \verb|PageLoadEvent|s gleichzeitig erhalten würde, aber immer nur eins davon an das CDP weitersenden kann. Dadurch würden gelegentlich Tabs unendlich auf ihr \verb|PageLoadEvent| warten. Um dies zu vermeiden haben wir ein einminütiges Timeout definiert, nach welchem wir annehmen, dass eine Seite geladen ist. \\ Schlussendlich speichern wir die vollständige Größe des gesamten Inhalts der Website ab. Dies ist nützlich, damit wir wissen, wie groß der Screenshot und die PDF sein müssen. Der Rückgabewert ist immer \verb|nil|, es sei den, bei der Navigation ist ein Fehler aufgetreten.

\subsection{Generierung von PDFs} \label{scraper:subsec:go:pdf}
Das Generieren von PDF Dateien nutzt die Druckansicht der Website. Effektiv wird also nur ein \quote{Rechtsklick > Drucken} simuliert. Diese Funktion ist Teil des Chrome Browsers und somit über das CDP Aufrufbar.  
\begin{listing}[h]
	\inputminted[
	frame=lines, 
	framesep=2mm, 
	linenos, 
	baselinestretch=1.2, 
	fontsize=\footnotesize]{go}{code/chromedp_pdf.go}
	\caption{Generieren von PDF Dateien}
	\label{scraper:code:pdf}
\end{listing}
Als einzige zusätzlich Aktion wird vorher die \verb|setDeviceMetricsAction| aufgerufen. Diese Funktion dient dem Zweck, die Maße des Tabs auf die Größe des gesamten Inhalts zu setzen, welche während der Navigation bestimmt worden war. Dies stellt sicher, dass die ganze Seite dargestellt wird. Das Ergebnis der \verb|PrintToPDF| Funktion ist die PDF, codiert in Base64. Wie diese Daten abgespeichert werden, besprechen wir in Abschnitt \ref{scraper:subsec:s3}.
\subsection{Generierung von Screenshots} \label{scraper:subsec:go:screenshot}
Um einen Screenshot zu Erstellen, benutzen wir den HTML Renderer von Chrome. Wenn eine Website vollständig geladen ist, wird sie automatisch von dem HTML Renderer verarbeitet. Nun müssen wir lediglich dessen Ausgabe abfangen und in ein Bildformat umwandeln. Das CDP unterstützt glücklicherweise eine solche Funktion mit dem \verb|CaptureScreenshot|. Wie bei dem Generieren einer PDF setzen wir lediglich die Maße des Tabs auf die Größe des Inhalts der Website und definieren zusätzlich den Viewport, also die dargestellte Größe, auf die gleichen Maße. Das Ergebnis ist ein Base64 - codiertes PNG Bild, welches wir wieder abspeichern können.
\begin{listing}[t]
	\inputminted[
	frame=lines, 
	framesep=2mm, 
	linenos, 
	baselinestretch=1.2, 
	fontsize=\footnotesize]{go}{code/chromedp_screenshot.go}
	\caption{Generieren von Screenshots}
	\label{scraper:code:screenshot}
\end{listing}















% !TeX encoding=utf8
% !TeX spellcheck = de-DE
\pagebreak
\section{Implementierung} \label{scraper:sec:implementation}
Mithilfe der oben besprochenen Implementierungen des CDP können wir nun anfangen einen mächtigen Service zu entwickeln, der in der Lage ist, eine Url entgegenzunehmen und für diese Screenshots und PDF Dateien zu erstellen. Zusätzlich definieren wir auch, dass dieser Service als einziger mit dem es-extractor zur Textentnahme kommuniziert. Folglich wird er in der Lage sein, eine Url entgegenzunehmen und dann als Antwort die kompletten Informationen über diese Website zu extrahieren und zurückzugeben. Wir geben ihm den Namen es-scraper. \\
Zuerst werden wir besprechen wie wir die generierten Screenshots und PDF Dateien abspeichern, danach sehen wir die Strukturen die sicherstellen, dass dieser Service so effizient wie möglich arbeitet und wie wir diese parallelen Abläufe kontrollieren.
\subsection{Speichern großer Dateien} \label{scraper:subsec:s3}
Für jede Website generieren wir einige relativ große Dateien. Elasticsearch ist zwar in der Lage, binäre Blobs abzuspeichern, zu viele von diesen verlangsamen aber das Generieren der Indizes. Zusätzlich wollten wir die Struktur der Dokumente in Elasticsearch nicht zu weit abändern. Stattdessen haben wir uns entschieden für die PDF Dateien, Screenshots und Thumbnails einen externen Objekt Speicher zu verwenden. \\
Dieser Objekt Speicher wird von der Library \verb|rook| gestellt. \verb|rook| ist ein Tool zur Speicherorchestrierung in Kubernetes. Es bietet viele verschiedene Möglichkeiten Daten abzuspeichern wie beispielsweise NFS, EdgeFS oder Ceph. Wir verwenden Ceph. Ceph ist ein verteiltes Dateisystem, welches über eine REST API gesteuert werden kann. Diese REST API basiert auf der API von Amazons S3 (Amazon Simple Storage Service). Deshalb können unsere Applikationen auch einfach mit bestehenden S3 Clients mit diesem verteilten Speicher kommunizieren. \\ \\
Die binären Daten können wir dann einfach hochladen, die Variable \verb|result| enthält die Daten die durch das CDP bestimmt worden:
\begin{minted}[
frame=lines, 
framesep=2mm, 
linenos, 
baselinestretch=1.2, 
fontsize=\footnotesize]{go}
// Save a screenshot
savePath, saverr := tab.Store.Upload(result, "png")
// Return an error to the user if the saving failed.
if saverr != nil { 
	ch <- ChromeTabReturns{nil, saverr}
	return
}
\end{minted}
Als Rückgabe erhalten wir eine Url, über die wir die Datei wieder herunterzuladen können. Diese Url speichern wir nun im Dokument in Elasticsearch.

\subsection{Parallele Bearbeitung von Tabs}
Die größte Optimierung, die wir vorgenommen haben, ist, dass eine Programminstanz theoretisch beliebig viele Tabs gleichzeitig bearbeiten kann. Dies ermöglicht, dass das System immer optimal ausgelastet ist. Hierfür haben wir eine Datenstruktur definiert, die einen Tab beschreibt. 
\begin{listing}[h]
	\inputminted[
	frame=lines, 
	framesep=2mm, 
	linenos, 
	baselinestretch=1.2, 
	fontsize=\footnotesize]{go}{code/es-scraper-tab.go}
	\caption{Tab – Datenstruktur}
	\label{scraper:code:tab}
\end{listing}
Die \verb|ID| ist lediglich eine Zahl die einen Tab eindeutig identifiziert und wird hauptsächlich verwendet um Protokolle lesbarer zu machen. Diese Zahl geht von $0$ bis Anzahl der Tabs$-1$. 
Wie in \autoref{scraper:sec:chromedp} besprochen, wird ein Tab intern im CDP durch einen Kontext beschrieben. Diesen Kontext nutzen wir nun aus, um Aktionen auf einem bestimmten Tab auszuführen. Die \verb|Stop| Funktion ist dazu da, um einen Tab zu schließen. Sie sollte aber generell nicht verwendet werden, da der Browser herunterfahren wird, wenn er erkennt, dass er weniger Tabs hat, als er anpreist. Die \verb|URL| beinhaltet die Website die momentan bearbeitet wird. Der \verb|State| beschreibt den Status, in dem sich dieser Tab befindet. Er kann entweder \verb|Accepting| oder \verb|Busy| sein. Wenn er \verb|Busy| ist, kann diesem Tab keine neue Aufgabe zugeordnet werden, da schon eine in Arbeit ist. \verb|Store| wird wie oben verwendet, um auf das verteilte Dateisystem zuzugreifen, das Screenshots, PDF Dateien und Thumbnails speichert. Der \verb|ContentSize| beinhaltet die absoluten Maße der aktuell geladenen Website. Diese werden verwendet, um zu bestimmen, wie groß der Screenshot und die PDF sein müssen um die gesamte Website einzufangen (siehe Abschnitte \ref{scraper:subsec:go:pdf}, \ref{scraper:subsec:go:screenshot}). Letztlich die \verb|MercuryURL| beinhaltet die URL, über die der Tab mit dem es-extractor kommunizieren kann. Der es-extractor bietet die Funktionalität den relevanten Text einer Website zu bestimmen. \\ \\ Zusätzlich zu der parallelen Bearbeitung von mehreren Tabs haben wir die einzelnen Funktionen so definiert, dass sie ebenfalls nicht aufeinander warten müssen. Auf diesem Weg können Screenshot, PDFs und der relevante Text einer Website gleichzeitig erzeugt werden. Als Aktionen auf einem Tab definieren wir einige Funktionen:

\paragraph{\mintinline{go}{func (tab *ChromeTab) Ready()}} Wird von der Aufgabenverteilung verwendet, um den Status eines Tabs auf \quote{bereit für die nächste Aufgabe} zu setzen.
\paragraph{\mintinline{go}{func (tab *ChromeTab) Busy()}} Wird von der Aufgabenverteilung verwendet, um den Status eines Tabs auf \quote{beschäftigt} zu setzen.
\paragraph{\mintinline{go}{func (tab *ChromeTab) Navigate(url string) error}} Navigiert einen Tab zu einer gegebenen URL und wartet dann darauf, dass diese Seite geladen ist (\autoref{scraper:code:navigate}). Speichert zusätzlich die Maße der Seite in \verb|ContentSize|. Läuft innerhalb eines Tabs nicht nebenläufig, sondern blockiert, bis die Seite geladen ist oder ein Fehler auftritt. 
\paragraph{\mintinline{go}{func (tab *ChromeTab) thumbnail(r *map[string]interface{}) (string, error)}} \text{ }\\ Nimmt den Rückgabewert des es-extractor (\autoref{extractor:table:mercury}) und lädt dann das Vorschaubild im Feld \verb|lead_image_url| herunter und speichert es. 
\paragraph{\mintinline{go}{func (tab *ChromeTab) setDeviceMetricsAction() chromedp.Action}}
Eine Hilfsfunktion, die eine \verb|chromedp.Action| erzeugt, mit der die Maße des Kontextes auf die absoluten Maße der Website in \verb|ContentSize| gesetzt werden kann.
\paragraph{\mintinline{go}{func (tab *ChromeTab) Screenshot(ch chan ChromeTabReturns)}} Erzeugt und speichert einen Screenshot der aktuell geladenen Seite. Schreibt dann ein JSON - Formatiertes Ergebnis mit der URL an der der Screenshot gespeichert wurde in den Kanal der beim Aufruf übergeben wurde. Falls es während der Abarbeitung zu einem Fehler kommt, wird dieser statt der Daten in den Kanal geschrieben. Ist in der Lage nebenläufig zu anderen Aktionen zu laufen.
\paragraph{\mintinline{go}{func (tab *ChromeTab) Pdf(ch chan ChromeTabReturns)}}  Erzeugt und speichert eine PDF Versoin der aktuelle geladenen Seite. Schreibt dann ein JSON - Formatiertes Ergebnis mit der URL an der die PDF gespeichert wurde in den Kanal der beim Aufruf übergeben wurde. Falls es während der Abarbeitung zu einem Fehler kommt, wird dieser statt der Daten in den Kanal geschrieben. Ist in der Lage nebenläufig zu anderen Aktionen zu laufen.
\paragraph{\mintinline{go}{func (tab *ChromeTab) Content(ch chan ChromeTabReturns)}} 
Schickt eine Anfrage an den es-extractor für die aktuell geladene Website den relevanten Text zu bestimmen. Lädt und speichert das Vorschaubild (Thumbnail). Ist in der Lage nebenläufig zu anderen Aktionen zu laufen.
\paragraph{\mintinline{go}{func (tab *ChromeTab) Scrape(ch chan ChromeTabReturns)}}
Kombiniert die Aktionen Screenshot, Pdf und Content um alle Informationen einer gegebenen Website zu extrahieren. Falls es während einer dieser Aktionen zu einem fatalen Fehler kommt, wird dieser zurückgegeben. Schreibt ein als JSON Formatiertes Ergebnis in den übergebenen Kanal. 
\begin{table}[h]
	\centering
	\begin{tabu}{lX}
		\toprule
		Feld & Beschreibung \\ \midrule
		\texttt{title} & Titel oder Überschrift der Website. \\
		\texttt{raw\_content} & Relevanter Text der Website, komplett ohne Textformatierung. Dieses Feld sollte genutzt werden um Suchanfragen und ähnliches zu starten. \\
		\texttt{markdown\_content} & Relevanter Text, der als Markdown \cite{markdown2016} formatiert ist. Dieses Feld sollte verwendet werden, um den Text anzuzeigen, da er die ursprüngliche Form des Artikels beibehält. Er kann auch Links zu Bildern enthalten, die heruntergeladen werden müssen. \\
		\texttt{author} & Autor oder Editor der Seite / Artikels. \\
		\texttt{date\_published} & Zeitpunkt zu dem die Website veröffentlicht wurde. \\
		\texttt{dek} & Abstrakte Zusammenfassung des Artikels, die unter dem Titel steht. \\
		\texttt{excerpt} & Ein kleiner Ausschnitt des Artikels, oft die ersten paar Sätze oder identisch mit dem \texttt{dek}. \\
		\texttt{total\_pages} & Anzahl der Seiten (HTML Dokumente) über die sich dieser Artikel erstreckt.\\
		\texttt{next\_page\_url} & URL der nächsten Seite (HTML Dokument) des Artikels, falls der Artikel aus mehreren Seiten besteht. \\
		\texttt{rendered\_pages} & Anzahl der Seite (HTML Dokumente), die betrachtet wurden. \\
		\texttt{url} & URL der Website. \\
		\texttt{word\_count} & Anzahl der Wörter im relevanten Text auf der Seite. \\
		\texttt{direction} & Vermutete Leserichtung des Artikels. Meistens \quote{ltr} oder \quote{rtl}. \\
		\texttt{thumbnail} & Downloadurl an der das Thumbnail heruntergeladen werden kann. Kann leer sein.\\
		\texttt{screenshot} & Downloadurl an der ein Screenshot der Seite heruntergeladen werden kann.  \\
		\texttt{pdf} & Downloadurl an der die PDF Version der Seite heruntergelanden werden kann.  \\
		\bottomrule
	\end{tabu}
	\caption{Rückgabewerte des es-scraper}
	\label{scraper:table:scraper}
\end{table}


\subsection{Die Auftragswarteschlange}
Eine Vielzahl von Tabs, die parallel diverse Aufgabe gleichzeitig bearbeiten benötigen eine Struktur, die sie kontrolliert und steuert. Um diese haben wir eine zusätzliche Auftragswarteschlange entwickelt, dieselbe Anzahl an Arbeiterthreads wie Tabs besitzt. \\
Bei dieser Auftragswarteschlange handelt es sich um eine Kanal mit einem Typ \verb|task|:
\begin{minted}[
frame=lines, 
framesep=2mm, 
linenos, 
baselinestretch=1.2, 
fontsize=\footnotesize]{go}
type task struct {
  Action   string
  URL      string
  Callback chan cdptab.ChromeTabReturns
}

type ChromeTabReturns struct {
  Data map[string]interface{} // Map containing unmarshaled json data.
  Err  error
}
\end{minted}
Ein \quote{Task} enthält also eine URL die zu bearbeiten ist und eine Aktion die auf dieser URL ausgeführt werden soll. Die Aktion wird durch den verwendeten Endpunkt bestimmt. Heißt, es könnte nur das Generieren einer PDF oder eines Screenshots sein, oder die Extraktion des relevanten Textes der URL oder es könnte ein vollständiger \quote{scrape} sein. Letzlich enthält eine Aufgabe, dann noch einen Kanal der von dem REST Interface erstellt wurde, als die Aufgabe in Auftrag gegeben wurde. Das REST Interface wartet dann auf ein beliebiges Datum in diesem Kanal (egal ob ein JSON Ergebnis oder ein Fehler) und schickt dieses Datum dann wieder an den Nutzer zurück und schließt die Verbindung. 

Bei der Initialisierung werden alle Tabs in eine Liste mit einem Mutex eingetragen. Da die Zuordnung der Tabs zu einer Aufgabe eine critical section ist, sorgen wir mit dem Mutex dafür, dass immer nur ein Tab gleichzeitig einer Aufgabe zugeordnet werden kann. Wenn eine Aufgabe in der Warteschlange ankommt, wird sie, falls ein Arbeiterthread frei ist, entgegengenommen. Ansonsten muss die Aufgabe warten, bis ein Thread bereit ist, sie entgegenzunehmen. Ein Thread mit einer Aufgabe wird dann ein freier Tab zugeordnet, auf dem der Thread die Aufgabe bearbeiten kann. Dazu wird der Zugang zu der Liste der Tabs gesperrt, der nächste freie Tab ausgewählt und dieser dann als \quote{beschäftigt} markiert. Dann wird der Zugang auf die Liste der Tabs wieder freigegeben und der Thread beginnt mit der Arbeit. Am Ende wird der Tab wieder freigegeben und der Thread ist bereit eine neue Aufgabe entgegenzunehmen. \\ Mit dieser relativ simplen Struktur können wir sicherstellen, dass Aufgaben dann bearbeitet werden, wenn Ressourcen frei sind.

\subsection{Schnittstellen}
Der es-scraper stellt 4 Endpunkte zur Verfügung, die alle nur auf einen HTTP 1.1 POST Request mit \verb|ContentType: application/json| reagieren. Alle erwarten ein JSON Dokument mit einem \quote{url} Feld: \mint{json}{{"url" : "http://example.com"}}
\paragraph{/scrape/content} Kommuniziert mit dem es-extractor um den relevanten Text der Website zu bestimmen und falls möglich, das Vorschaubild herunterzuladen. 
\paragraph{/scrape/pdf} Lädt die Website im CDP und erzeugt eine PDF Version die gespeichert wird. Eine URL zum Herunterladen der Datei wird dann in einem JSON Dokument zurückgegeben.
\paragraph{/scrape/screenshot} Lädt die Website im CDP und erzeugt einen Screenshot der gesamten Seite. Dieser wird abgespeichert und eine URL zum Herunterladen der Datei wird dann in einem JSON Dokument zurückgegeben.
\paragraph{/scrape/all} Kombiniert alle oberen Endpunkte und gibt ein JSON Dokument mit den Feldern in \autoref{scraper:table:scraper} zurück.
 
\subsection{Logging}
Der es-scraper besitzt umfangreiche Protokollierung die einem Administrator erlaubt den Programmverlauf einzusehen. Standardmäßig werden die Protkolle auf \verb|stdout| ausgegeben. Protokolleinträge beschreiben wann eine Aufgabe auf einem Tab bearbeitet wird. Alle Protokolleinträge, die während der Bearbeitung erzeugt werden beginnen mit der ID des Tabs in eckigen Klammern. Für alle 5 Aktionen (Navigieren, Vollständiger Scrape, PDF, Screenshot, Inhalt) werden Einträge erzeugt, wenn diese Angefangen und wenn sie abgeschlossen ist. Fehlt die Nachricht, dass eine der Aufgaben abgeschlossen wurde, kann man davon ausgehen, dass dieser Tab aus irgendeinem Grund eingefroren ist. Weiterhin gibt es Protokolleinträge, die beschreiben wann ein Ergebnis geschrieben wird. Wird laut den Protokolleinträgen ein Ergebnis geschrieben, aber kommt nicht an, kann dieses im Netzwerk untergegangen sein.

Wenn eine Instanz des es-scrapers im Kubernetes – Cluster läuft, sind die Protokolle
mit folgendem Kommando erreichbar, wobei xxxxxxxx-yyyyy die ID des Pods ist, für den
die Protokolle gewünscht sind. Alle existierenden Pods können mit \mintinline{bash}{kubectl get pods}
bestimmt werden.
\mint{bash}{kubectl logs es-scraper-deployment-xxxxxxxx-yyyyy es-scraper}
Zusätzlich kann man den Schalter \verb|-f| hinzufügen, um die Protokolle fortlaufend anzuzeigen.

\subsection{Fehlermeldungen}
Fehlermeldungen, die nicht fatal sind, werden protokolliert und die Aufgabe weitergeführt. Falls es zu einem fatalen Fehler kommt, wird dieser ebenfalls protokolliert und dann wird die Ausführung der aktuellen Aufgabe abgebrochen und der Nutzer bekommt eine HTTP 1.1 Antwort mit einem Statuscode ungleich 200. Falls möglich wird ebenfalls ein JSON Dokument mit einer deskriptiven Nachricht zurückgegeben.










% !TeX encoding=utf8
% !TeX spellcheck = de-DE
\section{Deployment} \label{scraper:sec:deployment}
Um eine beliebige Menge von Instanzen des es-scraper in das Kubernetes Cluster aufzusetzen, verwenden wir einen Service der den Port des Servers auf Port 80 umleitet, sowie ein Deployment, siehe \autoref{deployment:code:scraper}. Das Deployment definiert 6 benötigte Umgebungsvariablen, die Anzahl der gewünschten Replikate und das Docker Image. \\
\begin{table}[h]
\centering
\begin{tabu}{lX}
	\toprule
	Umgebungsvariable & Beschreibung \\ \midrule
	\texttt{S3\_ENDPOINT}& Endpunkt für der S3 Instanz\\
	\texttt{S3\_BUCKET\_NAME}& Name des Buckets der S3 Instanz\\
	\texttt{API\_BIND}& IP:PORT an dem der Server gestartet werden soll\\
	\texttt{MERCURY\_URL}& IP:PORT der es-extractor Instanz die verwendet werden soll, oder entsprechender Kubernetes Service\\
	\texttt{AWS\_ACCESS\_KEY\_ID} & Key ID für AWS\\
	\texttt{AWS\_SECRET\_ACCESS\_KEY} & Secret Key für AWS\\ \bottomrule
\end{tabu}
\caption{Umgebungsvariablen es-scraper}
\end{table}

\subsection*{Browserless}
Um Probleme mit verschiedene Versionen von Google Chrome zu vermeiden, verwenden wir zusätzlich das Docker Image browserless/chrome \cite{browserless}. Dies ist ein wohlgetester Browser der immer gleich funktioniert. Jede Instanz des es-scraper hat so ihren eigenen Browser, zu dem sie sich verbinden kann. Mit dem Kommando 
\mint{bash}{kubectl logs es-scraper-deployment-xxxxxxxx-yyyyy browserless}
kann der Log dieses Browsers eingesehen werden. Manchmal stürzen diese Browser ab und müssen manuell neu gestartet werden. Weiterhin kann man zusätzlich von diesem Image den Port weiterleiten, um auf ein Debugging Interface zuzugreifen.
