% !TeX encoding=utf8
% !TeX spellcheck = de-DE
\section{Entwicklung mit CDP in Go} \label{scraper:sec:chromedp}
Trotz der Zuverlässigkeit und Übersichtlichkeit von Google Chrome’s Puppeteer haben wir uns gegen die Verwendung dieser Library entschieden. Einerseits hatten wir keinen Entwickler im Team, der erfahren genug in der Verwendung von Node.js waren, um eine performante und stabile Implementierung zu entwickeln. Weiterhin haben wir in unseren ersten Tests Probleme mit der Nebenläufigkeit von Puppeteer gehabt. Wir konnten keinen Weg finden zuverlässig Screenshots und PDF Dateien einer Website gleichzeitig zu generieren. Da wir einen großen Fokus auf die Effizienz und Skalierbarkeit unserer Microservices legen, sind wir nicht in der Lage diese Library zu verwenden, obwohl es vermutlich die einfachste Variante ist. \\ \\
Die Programmiersprache Go bietet mächtige Primitive zur Entwicklung performanter, nebenläufiger Anwendungen. Es existiert ebenfalls eine direkte Implementierung des CDP in Go, bekannt als das Paket cdproto. Diese Library enthält generierten Code für alle des CDP definierten Befehle wie sie auch in der Referenzimplementierung zur Verfügung stehen. Diese Befehle zu generieren ist möglich, da das CDP nicht nur in Javascript implementiert ist, sondern auch vollständig mit PDL (Program Design Language) beschrieben ist. Aus dieser beschreibenden Sprache kann dann Quellcode für beliebige Programmiersprachen generiert werden.  \cite{cdprotogen}
Die Verwendung dieser Library ist immer noch sehr komplex. Auf der Grundlage des Pakets cdproto wird dann die Library chromedp entwickelt. Diese abstrahiert, ähnlich wie Puppeteer, die komplexen Funktionalitäten und vereinfacht die Nutzung. Ein großer Vorteil von der Implementierung in Go, dass die Standardbibliothek Kontexte ausgezeichnet unterstützt. \cite{go_context} Dadurch wird die Abbildung der Struktur einer Chrome Instanz sehr gut verwirklicht. Die Funktion \mint{go}{func NewContext(parent context.Context, opts ...ContextOption)} 
erstellt so je, nachdem mit welchem Vorgänger sie aufgerufen wird, einen Browser mit einem BrowserContext und einer Page oder einen BrowserContext mit einer Page in einem bestehenden Browser oder eine lediglich einen weiteren \quote{Tab} im selben BrowserContext.
\begin{listing}[h]
	\inputminted[
	frame=lines, 
	framesep=2mm, 
	linenos, 
	baselinestretch=1.2, 
	fontsize=\footnotesize]{go}{code/chromedp_context_1.go}
	\caption{Erstellung von geschachtelten Kontexten mit chromedp}
	\label{scraper:code:chromedpctx}
\end{listing}

\begin{listing}[h]
	\inputminted[
	frame=lines, 
	framesep=2mm, 
	linenos, 
	baselinestretch=1.2, 
	fontsize=\footnotesize]{go}{code/chromedp_context_2.go}
	\caption{Verwendung der Allocatorfunktionen in chromedp}
	\label{scraper:code:chromedpalloc}
\end{listing}

Wenn wir mehr Kontrolle über die Struktur unseres Browsers wünschen, können wir mit der Funktion \mint{go}{func NewExecAllocator(parent context.Context, opts ...ExecAllocatorOption)} nur den Allocator, also den Hauptprozess des Browsers, erzeugen. Mit diesem als Basis können wir nun beliebig viele BrowserContexts und Pages erstellen, indem wir wieder die Funktion \mintinline{go}{chromedp.NewContext(...)} verwenden. 


In dem Beispiel \ref{scraper:code:chromedpalloc} erschaffen wir nun also Schrittweise zuerst den Hauptprozess, dann zwei BrowserContext, welche man als \quote{Browser Fenster} verstehen kann. Gefolgt davon erzeugen wir 2 Tabs im ersten Fenster und einen Tab im zweiten Fenster. Wichtig ist, dass bei jedem Aufruf von \mintinline{go}{chromedp.NewContext(...)} immer ein Standardtab erzeugt wird, der Aktionen ausführen kann. Deshalb besitzen wir in Listing \ref{scraper:code:chromedpalloc} zwei Tabs mehr. Unsere Empfehlung ist, diese der Verständlichkeit wegen zu ignorieren. 

Um eine Aktion auf mit einem Tab auszuführen, dient die \\
\mintinline{go}{chromedp.Run(ctx context.Context, actions ...Action)} Funktion, die eine Liste von Aktionen auf einem Kontext ausführt. 
\begin{minted}[
frame=lines, 
framesep=2mm, 
linenos, 
baselinestretch=1.2, 
fontsize=\footnotesize]{go}
tasks := chromedp.Tasks{
  chromedp.Navigate("http://example.com"),
  chromedp.CaptureScreenshot(&result)
}
chromedp.Run(tab1_1, tasks)
\end{minted}
\pagebreak
\subsection{Verbindung zu einem bestehenden Browser} \label{scraper:subsec:go:remote}
Das CDP ermöglicht nicht nur die Erzeugung eines neuen Webbrowsers, sondern kann sich auch an einen bestehenden Browser verbinden. Dies hat den großen Vorteil, dass es zwischen verschiedenen Systemen keine unerwarteten Probleme gibt, wenn verschiedene Versionen von Chrome, den CDP und Go verwendet werden. Beispielsweise kann man einen Chrome Browser in einem Docker Container ausführen und so sicherstellen, dass auf allen Systeme derselbe Browser läuft. Die Funktion \mint{go}{func NewRemoteAllocator(parent context.Context, url string)} verbindet das CDP in Go mit einem beliebigen Chrome Instanz an einer Url. Es wird dann eine Verbindung über das Websocket Protokoll hergestellt. Damit sich das CDP an eine Browser Instanz verbinden kann, muss diese den entsprechenden Port öffnen. Mit dem Aufruf \mint{bash}{chrome --headless --remote-debuggig-port=9222} wird ein headless Browser gestartet, an den sich das CDP über  \verb|ws://localhost:9222| verbinden kann.

\subsection{Navigieren von Tabs} \label{scraper:subsec:go:navigate}
Um eine Funktion aus der chromedp Library, die nicht ausreichend oder fehlend ist, zu ersetzen verwendet man den Typ \mintinline{go}{type ActionFunc func(context.Context) error}, um eine herkömmliche Funktion in einem Kontext ausführbar zu machen. So haben wir die Navigation mit den Funktionen aus cdproto neu definiert, um sie an unseren Nutzen anzupassen.

\begin{listing}[h]
	\inputminted[
	frame=lines, 
	framesep=2mm, 
	linenos, 
	baselinestretch=1.2, 
	fontsize=\footnotesize]{go}{code/chromedp_navigate.go}
	\caption{Neudefinition der Navigation in chromedp}
	\label{scraper:code:navigate}
\end{listing}
Zuerst verwenden wir das \verb|emulation| Modul des CDP, um sicherzustellen, dass der User Agent gesetzt ist. Ohne User Agent bekommen wir keinen Zugang zu manchen Websites, die Beispielsweise erwarten, dass ein Nutzer zuerst den Cookies zustimmt, bevor man weitergeleitet wird. Daraufhin navigieren wir den Tab und warten darauf, dass ein \verb|PageLoadEvent| ankommt. Dieses Event signalisiert, dass die Website vollständig geladen ist und der Tab ihren gesamten Inhalt anzeigt. Wenn zu viele Seiten parallel geladen werden, kann es zu einem Data Race kommen. Dies liegt daran, dass der Browser mehrere \verb|PageLoadEvent|s gleichzeitig erhalten würde, aber immer nur eins davon an das CDP weitersenden kann. Dadurch würden gelegentlich Tabs unendlich auf ihr \verb|PageLoadEvent| warten. Um dies zu vermeiden haben wir ein einminütiges Timeout definiert, nach welchem wir annehmen, dass eine Seite geladen ist. \\ Schlussendlich speichern wir die vollständige Größe des gesamten Inhalts der Website ab. Dies ist nützlich, damit wir wissen, wie groß der Screenshot und die PDF sein müssen. Der Rückgabewert ist immer \verb|nil|, es sei den, bei der Navigation ist ein Fehler aufgetreten.

\subsection{Generierung von PDFs} \label{scraper:subsec:go:pdf}
Das Generieren von PDF Dateien nutzt die Druckansicht der Website. Effektiv wird also nur ein \quote{Rechtsklick > Drucken} simuliert. Diese Funktion ist Teil des Chrome Browsers und somit über das CDP Aufrufbar.  
\begin{listing}[h]
	\inputminted[
	frame=lines, 
	framesep=2mm, 
	linenos, 
	baselinestretch=1.2, 
	fontsize=\footnotesize]{go}{code/chromedp_pdf.go}
	\caption{Generieren von PDF Dateien}
	\label{scraper:code:pdf}
\end{listing}
Als einzige zusätzlich Aktion wird vorher die \verb|setDeviceMetricsAction| aufgerufen. Diese Funktion dient dem Zweck, die Maße des Tabs auf die Größe des gesamten Inhalts zu setzen, welche während der Navigation bestimmt worden war. Dies stellt sicher, dass die ganze Seite dargestellt wird. Das Ergebnis der \verb|PrintToPDF| Funktion ist die PDF, codiert in Base64. Wie diese Daten abgespeichert werden, besprechen wir in Abschnitt \ref{scraper:subsec:s3}.
\subsection{Generierung von Screenshots} \label{scraper:subsec:go:screenshot}
Um einen Screenshot zu Erstellen, benutzen wir den HTML Renderer von Chrome. Wenn eine Website vollständig geladen ist, wird sie automatisch von dem HTML Renderer verarbeitet. Nun müssen wir lediglich dessen Ausgabe abfangen und in ein Bildformat umwandeln. Das CDP unterstützt glücklicherweise eine solche Funktion mit dem \verb|CaptureScreenshot|. Wie bei dem Generieren einer PDF setzen wir lediglich die Maße des Tabs auf die Größe des Inhalts der Website und definieren zusätzlich den Viewport, also die dargestellte Größe, auf die gleichen Maße. Das Ergebnis ist ein Base64 - codiertes PNG Bild, welches wir wieder abspeichern können.
\begin{listing}[t]
	\inputminted[
	frame=lines, 
	framesep=2mm, 
	linenos, 
	baselinestretch=1.2, 
	fontsize=\footnotesize]{go}{code/chromedp_screenshot.go}
	\caption{Generieren von Screenshots}
	\label{scraper:code:screenshot}
\end{listing}














